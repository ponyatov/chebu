\secly{О журнале}

\subsecly{журнал хакспейса Контур}

\begin{itemize}[nosep]
\item обзоры студенческих работ
\item учебные материалы
\item публикации материалов по научной работе школьников
\end{itemize}

\subsecly{партнёры}

\begin{itemize}

\item Самарский университет
  \\\url{https://vk.com/samara_university}
  \\\url{https://ssau.ru/}

  \begin{itemize}

    \item Клуб любителей электроники "Контур"
    \\\url{https://vk.com/ssau_kontur}

  \item Студенческий робототехнический клуб "ROBOTIC"
    \\\url{https://ssau.ru/recreation/science/robot_tic}
    \\\url{https://vk.com/robotic_samara}

  \end{itemize}

  \item Самарский государственный технический университет
  \\\url{https://samgtu.ru/}
  \\\url{https://vk.com/samgtu_officiall}

  \begin{itemize}

    \item Клуб любителей электроники "Контур-Политех"
    \\\url{https://vk.com/samgtu_kontur}

\end{itemize}

\item Самарский региональный центр для одаренных детей "Вега"
  \\\url{https://codsamara.ru/}
  \\\url{https://vk.com/vegasamara_163}


\end{itemize}

\subsecly{ориентация на применение OpenSource}

при подготовке данных материалов ни один Microsoft не пострадал

\begin{description}
    \item[Linux] базовая операционная система, широко применяется как
    встраиваемая ОС для сложных устройств на базе микропроцессоров ARM и
    MIPS.\\Также любима и очень активно применяется как основная рабочая система
    профессиональными программистами, и учёными (для расчётов).
    \item[\TeX] система вёрстки научных изданий, работает в пакетном режиме:
    использует "компиляцию" текстовых файлов с собственной разметной.
\end{description}
