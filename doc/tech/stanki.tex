\secrel{Станочное оборудование}\label{stanki}
\secdown

\bigskip

Наиболее многочисленную группу металлорежущих станков составляют
\term{токарно-винторезные станки}, используются в механических, инструментальных
и ремонтных цехах заводов, а также в ремонтных мастерских в основном для
обработки деталей, имеющих форму тел вращения. При использовании соответствующей
оснастки позволяют растачивать отверстия в призматических (прямоугольных)
деталях, и фрезеровать небольшие детали. \emph{Самый ходовой тип детали\ ---
тела вращения с наружними и внутренними резьбами: валики, втулки, оси, болты,
винты, шпильки, кольца, шайбы и т.д.} К основным размерам, характеризующим
токарный станок, относятся
\begin{itemize}
  \item наибольший допустимый диаметр обрабатываемой заготовки,
  \item высота \term{центров} над станиной и
  \item расстояние между центрами.
  \item
Часто обращают внимание на диаметр \term{проходного отверстия шпинделя},
определяющий максимальный диаметр \term{длинномерных заготовок}, что важно при
изготовлении партий мелких деталей из длинных прутковых заготовок и нарезке
резьб на трубах.
\end{itemize}

\bigskip

Значительную часть среди металлорежущих станков составляют \term{фрезерные
станки}. Наибольшее распространение имеют консольно-фрезерные.
Предназначены для выполнения различных фрезерных работ цилиндрическими,
дисковыми, фасонными и другими \term{фрезами}, можно фрезеровать плоскости,
пазы, фасонные поверхности, и т.д. Кроме этого, универсальные
консольно-фрезерные станки c поворотным столом или делительной головкой
позволяют фрезеровать различного рода винтовые канавки и зубья зубчатых колес.
\emph{Для самодельной электроники интересны универсальные малогабаритные
фрезеры, способные работать в режимах вертикальной и горизонтальной фрезеровки,
для изготовления самых разнообразных деталей, а при установке в
горизонтальный шпиндель токарной оснастки и небольшую часть токарных работ}.

Основными размерами фрезерных станков, по которым можно определить возможность
установки и обработки конкретных заготовок с определенными габаритами, являются
размеры рабочей поверхности стола (длина и ширина) и \term{рабочий ход
стола}/\term{рабочая зона} в продольном, поперечном и вертикальном направлениях.
Этими размерами, и типом шпинделя, также определяется возможность установки
дополнительного оборудования, выпускаемого серийно: делительных столов,
расточных головок, оснастки для нарезки зубчатых колес и т.п.

\bigskip

Общая рекомендация\ --- берите самые большие станки с самой большой рабочей
зоной, какие можете себе позволить по цене, помещению для установки, мощностью
электропроводки, и стоимостью эксплуатации, обслуживания и расходных материлов.
Чем больше станок, тем б\'{о}льшую деталь вы сможете изготовить сами, и тем
больше возможностей по использованию дополнительного оборудования. Хотя
настольные станки дешевы и практически не требуют отдельного помещения, оснастку
для них (например поворотный столик) вы для них не найдете, придется ее делать
самому или где-то заказывать.

\input{tabletop/tabletop}

\input{stanki/selfmade}

\secrel{Промышленные станки}\secdown

Иногда хоббиту удается получить доступ к старым промышленным станкам. Наиболее
богаты в этом плане школы и другие учебные заведения, у них часто где-нибудь в
углу или подвале стоят пара старых станков призводства СССР. Несмотря на то, что
стоят они годами без движения, добраться до них получается с большим гемором:
нужно как минимум иметь официальный документ о наличии разряда токаря,
фрезеровщика или станочника широкого профиля. Кроме того, без получения
какого-то официального статуса, а соответственно и кучи ненужных обязанностей,
допуск к станку вы тоже не получите.

Общественых
\href{https://github.com/ponyatov/CHBZ/blob/master/presentation.pdf?raw=true}{открытых
технологических площадок} в Росcии не существует в принципе, большая часть
станочного оборудования установлена на закрытых территориях, или гниет в
подвалах и школах. 

\input{stanki/cccp}

\input{stanki/1A616}

\secup

\secup
