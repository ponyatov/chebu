\pagebreak
\secrel{$\mu$Станочная магия}

Для хоббита-паяльника, экспериментальных работ, и пилотных проектов часто
оказывается нужным изготавливать мелкие, иногда достаточно сложные детали для
крепежа, механики, разнообразные конструкционные элементы и особенно корпуса.

\medskip
Применение ЧПУ-станков для таких задач зачастую бессмысленно, так как они
подходят только для:
\begin{itemize}[nosep]
    \item (мелко)серийного производства
    \item изготовления деталей со сложными поверхностями, требующих
    одновременного движения инструмента по 2+ осям
    \item быстрого производства штучных, но ``библиотечных'', типовых деталей:
    поставил заготовку, выбрал готовую программу, и запустил нужное число раз.
\end{itemize}

\pagebreak
% \noindent
В практике, наоборот, чаще всего нужно изготовить несколько простых деталек,
типа втулочек или стоек для плат, зачастую подгоняя размеры ``по месту''. В
результате, сложность эксплуатации, цена и особенно стоимость
владения\note{техническое обслуживание, дорогие расходные материалы и
инструмент, необходимость обучения персонала, очень дорогой ремонт,..}\ ЧПУ
станком просто не оправдывается.

% \pagebreak
Кроме того, для ``хоббита'' важна шаговая доступность станка\ --- когда он стоит
в соседней комнате, и им всегда можно воспользоваться как только это окажется
нужным. Даже если у вас есть возможность бесплатно\note{а особенно если такой
возможности нет} отдать изготовление на ЧПУ, любой даже самый изношенный
токарничек оказывается крайне полезным. Вы просто встаёте, копаетесь в ящиках и
под столом, находите подходящий кусок прутка, и за минуты делаете всё что вам
было нужно, не тратя время на дорогу, подготовку модели/чертежа, поиски
свободного исполнителя,..

С этой точки зрения, даже изготовление самодельного настольного станочка на
уровне \term{действующего макета} оказывается оправданным. Естественно, вы не
получите той быстроты и удобства работы, как с настоящим станком. Однако,
применяя кое-какие приёмы, ручной инструмент, несложную оснастку, и имея
некоторый навык работы с такой ``жужжалкой'', вы вполне можете делать достаточно
сложные детали.

\begin{center}
\fig{tech/thumb1.png}{height=.55\textheight}
\fig{tech/thumb2.png}{height=.55\textheight}
\end{center}

\clearpage

\begin{wrapfigure}{r}{.27\textwidth}
    \fig{tech/ZitrekDP90.png}{width=.27\textwidth}
    \vspace{-1.2cm}
    % \caption{\tiny Zitrek DP-90}
\end{wrapfigure}

\noindent
Самые распространённые станки\ --- \term{сверлильные}, предназначены для
изготовления перпен\-ди\-ку\-лярно-круглых дырок\note{для максимально-дешёвых
станков китайской промышленности называть это ``отверстием'' кажется немного
подозрительным\ --- пиноль заметно стучит и болтает} в различных материалах.
Профессиональные модели иногда имеют функцию нарезания резьбы. Для монтажа
электроники и кустарного изготвления печатных плат часто используются очень
маленькие настольные сверлилки, чаще всего самодельные.

Исходя из максимальной нужности, и одновременно невысокой цены, даже для
хоббийной мастерской имеет смысл всё же \textit{купить} готовую самую дешёвую
китайскую ``сверлилку'' в настольном исполнении: у нас это
\href{https://www.youtube.com/watch?v=zEZ2bYbxiUo}{Zitrek DP-90}.

\pagebreak

\begin{wrapfigure}{l}{.3\textwidth}
    \fig{tech/2m112.png}{width=.3\textwidth}
    % \caption{2М112}
    \vspace{-1cm}
\end{wrapfigure}

\noindent
Если примерно оценить собственные усилия, затраты времени и метериалов, цена
такого станочка окажется ниже, чем его кустарное изготовление. При этом вы
получите ещё и поворотный столик, который позволяет сверлить отверстия под
углом. \term{Жёсткость станка}, качество изготовления, и особенно зазоры пиноли
конечно не сравняться с чем-то типа \emph{2М112}, но даже такого станка вполне
хватает сверлить сталь.

Как минимум, вы решите основную проблему при использовании \term{ручной дрели}\
--- \textit{сверление перпендикулярных отверстий}. Если купить или изготовить
\term{крестовые тиски}, на сверлилке можно попытаться
\href{https://www.youtube.com/watch?v=8kqSAwd3EVQ}{изобразить жалкое подобие}
\term{вертикально-фрезерного} станка, хотя бы черновую фрезеровку, c последующей
ручной доводкой.

\pagebreak
    
\noindent
Второй по нужности тип станков\ --- \term{токарные}, с функцией нарезания резьб.
Точёные детали очень интенсивно используются в конструкциях, особенно крепежные
элементы\note{винты, втулки, и т.п.}. К сожалению, цены на заводские варианты,
даже б/у, легко превышают стоимость приличного ноутбука.

\begin{wrapfigure}{l}{.22\textwidth}
    \fig{tech/sherline.png}{height=.6\textheight}
\end{wrapfigure}

\noindent
Из-за цены, заморачиваться с самодельными аналогами токарного станка уже
имеет смысл. Самым сложным узлом оказывается механизм зажима
детали, особенно его стыковка с валом шпинделя \term{передней бабки}.

\clearpage
\begin{wrapfigure}{r}{.22\textwidth}
    \fig{tech/spindleK01.png}{height=.4\textheight}
    \vspace{-1.5cm}
\end{wrapfigure}

Для самодельных станков на AliExpress доступен готовый узел шпинделя\ --- два
pillow-подшипника, шкив под ремень К-типа (?), и вал с резьбовой посадкой M14x1,
предназначенный для серийного патрона для микро-станков K01-50 или K01-63.

\begin{wrapfigure}{l}{.3\textwidth}
    \fig{tech/K01_63.png}{width=.3\textwidth}
\end{wrapfigure}

\vspace{1cm}
Трёхкулачковые токарные патроны из серии К01 для самоделок являются самым
удобным вариантом. Если не считать цену (как за настоящие станочные
комплектующие), недостаток который стоит отметить\ --- посадка М14х1 имеет
слишком маленький диаметр, чтобы можно было использовать сквозной канал через
шпиндель (для длинных прутков).

\clearpage
\begin{wrapfigure}{r}{.3\textwidth}
    \fig{tech/grinder.png}{width=.3\textwidth}
\end{wrapfigure}

Для черновой обработки твёрдых материалов применяются различные варианты
шлифовки, в частности\ --- \term{ленточный гриндер}. Замкнутую кольцом ленту
наждачной ткани раскручивают на роликах, в рабочей зоне она опирается на
\textit{плоскую жёсткую площадку}, что позволяет шлифовать плоскости.

Конструкция гриндеров достаточно разнообразна, и может меняться и дополняться
оснасткой для выполнения множества задач, заточки инструмента, или выполнения
специализированных операций\note{например получение спусков и поверхностей при
ихготовлении ножей}. С точки зрения применения в электронике, например, можно
сделать гриндер для обработки стеклотекстолита для печатных плат, и полировки
меди перед нанесением фоторезиста.
