\secrel{Linux \& OpenSource}\label{linux}\secdown

\clearpage
\secrel{Ориентация на применение OpenSource}

при подготовке материалов журнала ни один Microsoft не пострадал

\begin{description}

    \item[Linux] базовая операционная система, широко применяется как
    встраиваемая ОС для сложных устройств на базе микропроцессоров ARM и
    MIPS.\\Также любима и очень активно применяется как основная рабочая система
    профессиональными программистами, и учёными (для расчётов).

    \item[\TeX] система вёрстки научных изданий, работает в пакетном режиме:
    использует "компиляцию" текстовых файлов с собственной разметной.

    \item[KiCAD] специализированный САПР для работы со схемами и трассировки
    печатных плат

    \item[FreeCAD] САПР общего назначения

    \item[Maxima] система компьютерной алгебры (символьные вычисления)
    
    \item[VSCode] IDE распространяемая по беслатной лицензии MIT

    \item[средства разработки ПО]\ \\используются только свободные/бесплатные
    реализации\\ языков программирования, и библиотек (GNU gcc, Python, Rust,..)

\end{description}

% \clearpage
\noindent
использование бесплатных версий коммерческих продуктов не рекомендуется для
избежания vendor lock и необоснованных претензий:
\begin{itemize}[nosep]
\item смена политики распространения ПО и условий лицензирования по желанию
компаний-производителей, например как это случилось с Oracle JDK
\item введение санкций для "тяжёлых"\ и узкоспециализированных САПР
\item слежка и сбор данных, рассылка исков при обнаружении принесённого
студентами "левого"\ ПО и открытия проектов, сделанных в ломаных программах.
\end{itemize}

\bigskip\noindent
для проектов предпочтителен подход OpenSource \& OpenHardware:
\begin{itemize}[nosep]
    \item свободная публикация, повторение, и модификация
    \item стимулирование обмена знаниями и навыками
    \item экономия усилий за счёт создания открытых бесплатных библиотек
    \item избавиться от затрат на поддержку проектов: исправить ошибки, сделать
    доработки, и опубликовать исправления может любой желающий
    \item защита от патентных троллей: опубликованные идеи и проекты не могут
    быть захвачены через их патентование сторонними лицами
\end{itemize}

\secup
