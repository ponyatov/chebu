\clearpage
\secrel{\vbx: как ничего не сломать?}

Для разработки ПО, особенно системного, без риска повредить ОС и настройки вашей
рабочей системы, необходимо применение \term{виртуализации}: системы типа \vbx\
позволяют вам эмулировать отдельные компьютеры, и даже создавать из нескольких
виртуальных компьюеров сети.

\bigskip\noindent
\href{https://www.youtube.com/watch?v=j1FAZ0bUEvs&list=PLddc343N7Yqi3Uoj0gk0d-2C2ifhcnQ7Y&index=5}{несколько видео-уроков} 
по установке \vbx\ и \term{гостевых ОС} \youtube

\bigskip
Использование \vbx\ позволит вам неограниченно экспериментировать с любыми ОС
для платформ x86/amd64\note{даже писать свою низкоуровневую ОС, или драйвера}.
Максимум, чего вы можете добиться неумелыми действиями, или кодом с ошибками\
--- придётся пересоздать виртуальную машину, но при этом на вашем рабочем
компьютере всё будет работать как и раньше.
