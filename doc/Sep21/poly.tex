\secrel{Новости площадок}\secdown

\secrel{Контур-Политех}

Рады сообщить о создании новой "контурной" команды, имя которой\ ---
"Контур-Политех"! Как понятно из названия, базироваться данное подразделение
клуба будет в Самарском Политехе, с привязкой к инженерно-технологическому
факультету; руководителем же станет прекрасная
\href{https://vk.com/kkkshch}{Виктория}, аспирантка ИТФ.

\bigskip\noindent
открывается новая "Зона 51":
\begin{itemize}[nosep]
\item в 7 корпусе (Первомайская, 1)
\item в 758 кабинете (3 этаж)
\end{itemize}

\bigskip
\noindent\fig{Sep21/dipcat.jpg}{height=.5\textheight}

\clearpage
\secrel{Контур}

Объявлены победители и призеры Всероссийского конкурса технологических кружков.
Самарский университет им. Королёва вошел в число призеров конкурса!

\bigskip\noindent
Победу одержали:
\begin{itemize}[nosep]
\item Центр беспилотных систем
\item Молодежная аэрокосмическая школа
\item Клуб любителей электроники "Контур"
\end{itemize}

\bigskip\noindent
Подробнее: \url{https://ssau.ru/news/19504}

\bigskip\noindent
Участники конкурса
размещены на интерактивной карте технологических кружков России
\url{https://map.kruzhok.org/}.

\clearpage\noindent
"Сегодня технологические кружки – это не просто центры дополнительного
образования, кружки становятся средой выявления и развития талантов и,
одновременно, местом зарождения новых технологических проектов и стартапов.", –
отметила заместитель министра науки и высшего образования РФ Елена Дружинина.

\secrel{Центр Вега}


\secup
