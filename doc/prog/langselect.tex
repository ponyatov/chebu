\secrel{Выбор языка программирования для проектов}

Для быстрой и удобной разработки программ у нас есть достаточно богатый выбор:

\begin{description}
    
\item[\py] универсальный язык общего назначения,
    \begin{itemize}[nosep]
        \item лёгкий в освоении, высокоуровневый\note{не нужно заморачиваться с
        выделением памяти, прописывать типы, ловить исключения с указателями,..}
        \item синтаксис \textit{заставляет}\ писать читаемый код\note{табуляции
        рулят \Smiley, не забываем про \prog{autopep8}},
        \item очень богатый набор готовых библиотек на все случаи
    \end{itemize}

\item[\js] = JavaScript (не путать с \java !!)\\
    язык для Web-разработки, из-за своих странностей и нелогичностей зачастую
    вызывает желание взять и чем-нибудь долбануть. Возможно это связано с тем,
    что он интенсивно использовался веб-дизайнерами\note{и на язык оказал
    сильное влияние гуманитарный склад ума его пользователей, и видимо
    разработчиков самого языка:
    \href{https://www.youtube.com/watch?v=et8xNAc2ic8}{отсутствие проверки типов
    и прочий WAT?!}} последние пару десятков лет.    
    Основное достоинство языка конечно общедоступность: переключитеcь в браузер,
    нажмите \menu{\keys{F12}>Console}... упс, вы уже \Smiley. Не нужно ничего
    устанавливать, доступен на любом устройстве и в любом браузере, для того
    чтобы начать писать код нужно только создать пару файлов и ``блокнот''.

    Из-за своей распространённости \js\ пытаются использовать для задач
    автоматизации, и даже пытаются засунуть в микроконтроллеры, но результат
    предсказуем. Можно отметить систему
    \href{https://habr.com/ru/company/yandex/blog/519600/}{Node-RED}\ ---
    визуальное \term{потоковое программирование} для систем сбора данных, IoT,
    ``умного дома'', и мелкой автоматизации.

\item{\wasm}    

    Ходят разговоры о замене в браузерах интерпретатора JS на более
    низкоуровневую \term{WASM}-машину, и возможность использовать для
    веб-разработки любые языки, но особых подвижек в этом направлении не видно.
    Из-за отсутствия прямого доступа к DOM-дереву веб-страницы приходится
    работать через \js-прослойку, и применение \wasm\ пока ограничивается только
    вычислительными задачами\note{имеется в виду конечно front-майнинг \Smiley,
    другие примеры нужно сильно поискать}.

\item[\java]

    язык-стандарт для разработки тяжёлых корпоративных и финасовых систем:
    банковское ПО, биллинг сотовых операторов, системы управления
    производством,..

    За счёт JIT-компиляции программы работают действительно быстро (когда
    разгонятся), хороший жёсткий контроль типов, но рантайм языка крайне тяжёл\
    --- для работы программ требует сотни мегабайт и гагабайты ОЗУ, запуск
    виртуальной машины и более-менее тяжёлого приложения легко просаживает самый
    мощный компьютер на десятки секунд.

% \clearpage
\item[\cs]

    очень популярный язык на платформе \win/\net
    
    Часто используется в научной среде для расчётов и создания различного
    вспомогательного ПО, но лучше его не использовать: язык слишком сильно
    завязан на Micro\$oft, получаете необоснованный vendor и platform lock
    только потому что поленились взять другую книжку.

    К сожалению, вы вполне можете с ним столкнуться при автоматизации
    производства: зачем-то на нём пишут SCADA, системы АСУ и САПР. В сообществе
    \os\ есть определённые подвижки по созданию альтернативной открытой
    платформы $\mathbf{Mono}$, но по факту всё практически-значимое ПО наквозь
    пронизано vendorlock-зависи\-мо\-стя\-ми, и прибито толстенными гвоздями к
    \win.
\end{description}

\paragraph{Языки системного и низкоуровневого программирования}

\begin{description}

\item[\cb] классическая ``сишечка''

    \begin{itemize}[nosep]
    \item нужен всегда и везде, особенно при написании прошивок для
    микроконтроллеров;
    \item язык низкого уровня, максимально близок к ``железу'', поэтому
    позволяет максимально оптимизировать код по скорости и памяти, ценой
    вырванных волос и сотен часов отладки;
    \item язык нерасширяем, средства абстракции практически отсутствуют\ ---
    гарантированы незабываемые месяцы долботни кода, вылеты по системным
    ошибкам, звёздочки двойных указателей в глазах, и вечная отладка.
    \end{itemize}

\item[\cpp] главный промышленный язык, когда нужно быстро и много

    Отличают ``Си с классами'', и ``настоящий \cpp'' когда вы способны
    продраться через дискретную алгебру в книгах Степанова, и хоть что-то
    понять\dots\ срок обучения\ --- годы.

    Очень сложный язык, необходимость ручного управления памятью, множество
    трудно уловимых косяков при использовании многопоточности, закат Cолнца
    вручную\dots\ Программисты вынуждены мириться со сложностью, и нарабатывать
    автоматизм при написании кода, потому что языков-альтернатив промышленного
    уровня для компактных и максимально быстрых программ практически нет:
    никакая \java\ не влезет в 20К ОЗУ, а \rust\ ещё сыроват\note{но
    перспективен для решения именно проблем C/\cpp\ с памятью и многопоточкой}.

    Язык \cpp\ наиболее универсален, работает везде от пищалок c единицами
    килобайт ОЗУ до тяжёлых серверов и супер-кластеров, при некотором желании и
    извращениях можно писать не только серверный код для web, но даже части
    frontend используя компилятор \prog{clang}\ и
    \href{https://emscripten.org/}{Emscripten}.

% \clearpage
\item[\rust] \ref{rust} процедурно-функциональный язык системного и
многопоточного программирования (высоконагруженные веб-серверы, крипта/финтех,
интерпретаторы/компиляторы)

    Язык расширяем, с поддержкой полноценных макросов, в комплексе с жёстким
    контролем типов и обращений к памяти в параллельном коде даёт большие шансы
    на полную замену \cb, с претензией на \cpp.

    \rust\ применяется для приложений, критичных к безопасности\ --- финтех,
    блохчейн, программы ИТ-аудита и контроля. Постепенно распространяется как
    системный язык общего назчения, в т.ч. начал применяться для написания
    специализированных ОС, и драйверов Linux.

    Распространение языка в основном сдерживают две вещи:
    \begin{enumerate}[nosep]
        \item сложность изучения, и понимания владения данными и move-семантики;
        \item отсутствие наследования в ООП, язык скорее процедурный, вместо
        классов используются трейты, что-то типа интерфейсов в \java.
    \end{enumerate}

\item[\F] \ref{rdw} специфический командный язык, с очень примитивным
синтаксисом, и минимальными требованиями к выч.ресурсам
    \begin{itemize}[nosep]
    \item хорош как командная оболочка для встраиваемых систем;
    \item с некоторыми оговорками может использоваться как скриптовый язык
    общего назначения (конфигурационные файлы, встраиваемый скриптовый движок
    для программ);
    \item главное достинство\ --- крайняя простота, работающий интерпретатор
    пишется за пару недель любым начинающим программистом;
    \end{itemize}

\end{description}

% \pagebreak
Если вы думаете, что сможете отвертеться, и будете писать только прошивки для
микроконтроллеров, или заниматься исключительно обработкой данных, вы
огорчитесь\ --- \textit{пара-тройка языков сейчас входит в обязательный комплект
любого инженера}, и их в любом случае придётся осваивать хотя бы на минимальном
уровне:
\begin{enumerate}%[nosep]
    \item HTML/\js/CSS\ --- интерфейс пользователя, печатные формы, ввод данных,
    визуализация, клиент для мобильних устройств
    \item SQL\ --- хранение и простая обработка данных
    \item \py\ --- сложная обработка данных, \term{прототипирование},
    desktop-клиенты со сложным GUI
\end{enumerate}
